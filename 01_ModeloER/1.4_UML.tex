\documentclass[t]{beamer}

% Load general definitions
\input{preamble.tex}

% Specific definitions
\title[]{Banco de dados II}
\subtitle[]{Diagramas de classes UML}
\author[]{Patrícia Lucas\\{\footnotesize }}
\institute{Bacharelado em Sistemas de Informação \\ IFNMG  - Campus Salinas}
\date{\scriptsize Salinas\\Março 2021}

\begin{document}

% cover page
\setbeamertemplate{footline}{}
\begin{frame}

\begin{center}
\includegraphics[width=.15\textwidth]{}
\end{center}
  \titlepage
  \begin{tikzpicture}[remember picture,overlay]
  \node[anchor=south east,xshift=-5pt,yshift=5pt] at (current page.south east) {\tiny Versão 1.2021};
  \node[anchor=south west,yshift=0pt] at (current page.south west) {\includegraphics[width=.25\textwidth]{Logos/salinas_horizontal_jpg.jpg}};
  \end{tikzpicture}  
\end{frame}

% Main slides

\begin{ftst}{Referência}{Diagramas de classes UML}
\begin{figure}
    \centering
    \includegraphics[scale=0.4]{Figuras/book.jpg}
\end{figure}
ELMASRI, R.; NAVATHE, S. B. Sistemas de Banco de Dados. 7. ed. São Paulo: Pearson Addison Wesley, 2019.
\end{ftst}

%==================================

\begin{ftst}{Visão geral}{Diagramas de classes UML}

\begin{itemize}
    \item \textbf{UML: } Linguagem de Modelagem Unificada
    \item \textbf{Uso: } é uma linguagem-padrão para a elaboração da estrutura de projetos de software.
    \item \textbf{Diagramas: } de classe, de casos de uso, de estados, de atividades, de sequência...
    \item \textbf{Os diagramas de classes podem ser considerados uma notação alternativa aos diagramas ER.}
\end{itemize}
\end{ftst}

%==================================


\begin{ftst}{ER para UML}{Diagramas de classes UML}

Vamos traduzir o modelo ER para UML:

\begin{figure}
    \centering
    \includegraphics[scale=0.1]{Figuras/03_02.png}

\end{figure}
\end{ftst}

%==================================

\begin{ftst}{ER para UML}{Diagramas de classes UML}

As entidades são modeladas como classes:
\vone
\begin{figure}
    \centering
    \includegraphics[scale=0.095]{Figuras/UML_1.png}

\end{figure}
\vone
\small
*Não é necessário especificar o atributo chave.
\end{ftst}

%==================================

\begin{ftst}{ER para UML}{Diagramas de classes UML}

Os atributos multivalorados são modelados como classes também:
\vone
\begin{figure}
    \centering
    \includegraphics[scale=0.15]{Figuras/UML_2.png}

\end{figure}
\end{ftst}

%==================================

\begin{ftst}{ER para UML}{Diagramas de classes UML}
\begin{itemize}
    \item Relacionamentos em UML são chamados de associações.
    \item Um relacionamento binário é modelado como uma associação:
\end{itemize}

\vone
\begin{figure}
    \centering
    \includegraphics[scale=0.15]{Figuras/UML_3.png}

\end{figure}
\end{ftst}

%==================================

\begin{ftst}{ER para UML}{Diagramas de classes UML}

Relacionamento com atributos são modelados como classes associativas:

\vone
\begin{figure}
    \centering
    \includegraphics[scale=0.1]{Figuras/UML_4.png}

\end{figure}
\end{ftst}

%==================================

\begin{ftst}{ER para UML}{Diagramas de classes UML}

Autorrelacionamentos são modelados como associações reflexivas:

\vone
\begin{figure}
    \centering
    \includegraphics[scale=0.12]{Figuras/UML_5.png}

\end{figure}
\end{ftst}

%==================================

\begin{ftst}{ER para UML}{Diagramas de classes UML}

Relacionamentos identificadores são modelados como composições:

\vone
\begin{figure}
    \centering
    \includegraphics[scale=0.11]{Figuras/UML_6.png}

\end{figure}
\end{ftst}

%==================================

\begin{ftst}{ER para UML}{Diagramas de classes UML}

Relacionamentos ternários são modelados como associações ternárias:

\vone
\begin{figure}
    \centering
    \includegraphics[scale=0.12]{Figuras/UML_7.png}

\end{figure}
\end{ftst}

%==================================

\begin{ftst}{ER para UML}{Diagramas de classes UML}

Especialização/generalização é modelada como uma generalização:

\vone
\begin{figure}
    \centering
    \includegraphics[scale=0.12]{Figuras/UML_8.png}

\end{figure}
\end{ftst}

%==================================

\begin{ftst}{ER para UML}{Diagramas de classes UML}

Especialização/generalização total é modelada como uma classe abstrata:

\vone
\begin{figure}
    \centering
    \includegraphics[scale=0.12]{Figuras/UML_9.png}

\end{figure}
\end{ftst}

%==================================

\end{document}